\documentclass[a4paper]{article}
\usepackage[utf8]{inputenc}
\usepackage[spanish]{babel}
\usepackage{datetime}
\usepackage{hyperref}

% https://tex.stackexchange.com/a/212264/273585
\newdateformat{monthyeardate}{\monthname[\THEMONTH] de \THEYEAR}

\begin{document}

\thispagestyle{empty}  % prevent page number on first page
                        % https://latex.org/forum/viewtopic.php?t=25462

Francisco Figari - computador científico radicado en Buenos Aires

\textit{Curriculum Vitae} - \monthyeardate\today

\url{http://ffig.ar/i}

\begin{itemize}

  \item Experiencia
  \begin{itemize}
    \item
      Desarrollador backend @ \textit{Brubank} desde mayo 2023. \newline
      Extensión y construcción de sistemas backend orientados a comunicar
      servicios internos, servicios externos e interfaces \textit{graphql}.

    \item
      Pasante de investigación @ \textit{DC, FCEyN, UBA} entre agosto 2021 y
      septiembre 2022 bajo programa de becas para pasantías de iniciación en
      la investigación. \newline
      Estudio sobre el uso eye tracking en navegadores web aplicado a análisis
      clínicos a distancia, realizado en torno a tesis de licenciatura.

    \item
      Programador fullstack @ \textit{DubbingDigital} entre marzo 2021 y
      diciembre 2021.  \newline
      Extensión y mantenimiento de interfaz y servidor web de software de
      gestión.

    \item
      Programador lead @ \textit{UrbanSim} entre julio 2018 y noviembre 2020.
      \newline
      Inicialmente extensión de generador procedural de vivienda asequible
      multifamiliar luego desarrollo de plataforma web envolvente.

    \item
      Pasante de investigación @ \textit{ARSLab, Carleton University} entre
      febrero y junio de 2018. \newline
      Estudio sobre la aplicabilidad de simulación al desarrollo de
      \textit{firmware} de sistemas embebidos.

    \item
      Ayudante de cátedra en la materia Métodos Numéricos @ \textit{DC, FCEyN,
      UBA}.  \newline
      4 cuatrimestres ejercidos entre 2016 y 2018 distribuidos entre la
      práctica y el laboratorio.

    \item
      Programador \textit{freelance} @ \textit{LIAA, FCEyN, UBA} en fin de 2017.
      \newline
      Exploración del uso de \textit{eye tracking} al estudio de atención en
      navegadores web.

    \item
      Programador \textit{junior} @ \textit{Eryx} durante 4 meses a principios
      de 2016.

    \item
      Ayudante de cocina y mozo @ \textit{La Capelina, San Fernando} durante
      veranos 2012 y 2013.

  \end{itemize}

  \item \textit{Skills} laborales
  \begin{itemize}

    \item
      Experiencia profesional con múltiples lenguajes (\texttt{Golang},
      \texttt{Javascript}, \texttt{C}\verb|++|, \texttt{Python}) y paradigmas
      (objetos, eventos, procedural). \newline
      Buscando aprender nuevos estilos y aplicaciones de la programación.

    \item
      Adopción de prácticas \textit{agile} y metodologías \textit{kanban}.
      Uso de \textit{Jira}. \newline
      Experiencia trabajando en equipos interdisciplinarios conformados por
      desarrolladores \textit{backend}, \textit{mobile} y \textit{frontend},
      managers, \textit{qa}, \textit{data scientists}, urbanistas e
      investigadores, lideando con proveedores y liderando pequeños equipos de
      desarrollo (entre 2 y 3 desarrolladores).

    \item
      Habilidad para construir interfaces web y \textit{web api}s
      (\texttt{Bootstrap}, \newline \texttt{ExpressJS}, \texttt{Flask}).
      \newline
      Comprensión intermedia de la API \texttt{JavaScript} para interactuar con
      el navegador. \newline
      Experiencia con librerías y \textit{frameworks} reactivos de
      \textit{frontend} (\texttt{ReactJS}, \texttt{VueJS},
      \texttt{Redux}).\newline
      Conocimientos elementales sobre cómo levantar, mantener y migrar bases de
      datos relacionales (\texttt{PostgreSQL}).

    \item
      Experiencia trabajando con distintos códigos de planeamiento urbano y
      distintos sistemas de coordenadas.
      Manipulación de objetos geoespaciales: dibujado en \textit{frontend}
      (\texttt{MapBox}, \texttt{TurfJS}); almacenamiento (\texttt{PostGIS});
      manipulación en algoritmos. \newline
      Uso de APIs de geocodificación (\texttt{GoogleMaps Geocoding API}).

    \item
      Interés en aplicar y estudiar buenas prácticas de programación y
      desarrollo de software. \newline
      Conocimiento general de sistemas \texttt{Unix} (\texttt{git},
      \texttt{ssh}, \texttt{bash}).
      Conocimiento básico de \textit{DevOps} (\texttt{webpack},
      \texttt{kubernetes}, \texttt{gcp}).

    \item
      Inglés (uso diario a través de estudio y ocio) y Francés (escolaridad).

  \end{itemize}

  \item Estudios
  \begin{itemize}

    \item
      Licenciatura en Ciencias de la Computación @ \textit{DC, FCEyN, UBA},
      iniciada en 2013 y finalizada en 2022.

    \item
      Secundario @ \textit{Licée Jean Mermoz, Buenos Aires}, finalizado en
      2013.

  \end{itemize}

\end{itemize}

\end{document}
